\documentclass[12pt]{article}
\usepackage{amsmath}
\usepackage[T1]{fontenc}
\usepackage[utf8x]{inputenc}

\title{How to do partial elimination with BLAS and LAPACK}
\begin{document}
\maketitle

\section{Problem statement}
Lets assume that we have equation system of size $n$. We want to partially eliminate first $m$ variables from the system, so that $k = n - m$ variables may still be not eliminated. What we need to do is to calculate Shur complement of the equation system.
\begin{equation}
    \begin{bmatrix}
    A_{m × m} & B_{m × k} \\
    C_{k × m} & D_{k × k}
    \end{bmatrix}_{n × n} x = \begin{bmatrix}
    E_{m × 1} \\
    F_{m × 1} 
    \end{bmatrix}_{n × 1}
\longrightarrow
    \begin{bmatrix}
    \boldsymbol{I}_{m × m} & B'_{m × k} \\
    \boldsymbol{0}_{k × m} & D'_{k × k}
    \end{bmatrix}_{n × n} x = \begin{bmatrix}
    E'_{m × 1} \\
    F'_{m × 1} 
    \end{bmatrix}_{n × 1}
\end{equation}

\section{Algorithm}
\begin{align}
\label{step1}    B' &= A^{-1} \times B \\ 
\label{step2}    E' &= A^{-1} \times E \\
\label{step3}    D' &= D - C \times B' \\
\label{step4}    F' &= F - C \times E'
\end{align}


\section{BLAS/LAPACK implementation}

Let's assume that
\begin{itemize}
    \item matrices have continuous columns (like in Foratran), which is not standard behavior in C,
    \item \textbf{G} is LHS matrix of the system that is composed of submatrices \textbf{A} to \textbf{D},
    \item \textbf{H} is RHS vector of the system that is composed of subvectors \textbf{E} and \textbf{F},
    \item \textbf{A} to \textbf{F} are pointers to the begining of the submatrices,
    \item \textbf{n}, \textbf{m} and \textbf{k} are defined as in introduction,
    \item we check if operation succeeded after operations that may fail (e.g. LU decomposition),
    \item the operation will be in-place and will modify input matrix.
\end{itemize}

\paragraph{Algorithm}
\begin{enumerate}
    \item Do the LU decomposition for submatrix A: \\ \texttt{dgetrf(m, m, A, n, ipiv, status)} \\ The row pivot vector returned in \texttt{ipiv} will be needed in next two steps.
    \item $B = (PLU)^{-1} \times B$: \\ \texttt{dgetrs(NoTrans, m, k, A, n, ipiv, B, n, status)}
    \item $E = (PLU)^{-1} \times E$: \\ \texttt{dgetrs(NoTrans, m, 1, A, n, ipiv, E, n, status)}
    \item $D = D - C \times B$: \\ \texttt{dgemm(NoTrans, NoTrans, k, k, m, -1.0, C, n, B, n, 1.0, D, n)}
    \item $F = F - C \times E$: \\ \texttt{dgemv(NoTrans, k, m, -1.0, C, n, E, 1, 1.0, F, 1)}
    \item\label{diag} $A = \boldsymbol{I}_{m × m}$ – $A$ is identity matrix
    \item\label{zero} $C = \boldsymbol{0}_{k × m}$ – $C$ is zero matrix
\end{enumerate}

Steps \ref{diag} and \ref{zero} may be omited if you only need Schur complement (matrices D and F).

\end{document}
